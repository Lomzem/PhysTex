\documentclass[fleqn]{article}
\setlength{\parskip}{\baselineskip}%
\setlength{\parindent}{0pt}%

\usepackage[table]{xcolor}
\usepackage{siunitx}
\usepackage{tabularx}
\usepackage{float}
\usepackage{amsmath}

\begin{document}
\setlength{\mathindent}{0pt}

\section*{Spring Potential Energy}
Spring potential energy is a \underline{restorative force} (it returns to equilibrium)


\[ \text{Hookes Law: } \vec{F}_{s}  = -k \Delta \vec{x} \]
\[ \Delta U=-W \]
\[ \Delta U _{s} = - W _{s}  \]


\[ W = \int_{A}^{B} \vec{F} _{s} \ d \vec{r}  \]

\[ \vec{F} _{s} \text{ in the direction } \leftarrow \]
\[ \Delta \vec{r} \text{ in the direction } \rightarrow \]
\[ \angle \text{ between the vectors is }180 ^{\circ} \]

\[ W _{s} = \int_{A}^{B} |\vec{F} _{s} | |d \vec{r} |\ \cos(180 ^{\circ}) \]
\[ W  _{s} = - \int_{A}^{B} k \Delta x \ dx\]
\[ W _{s} = \left[ -\frac{1}{2} k \Delta x ^2 \right]^B_A \]

\[ \text{ Remember: } \Delta U _{s} = -W _{s}   \]
\[ U _{sA} = \frac{1}{2} k (\Delta x _{A}) ^2 \]

\[ U _{s} = \frac{1}{2} k (\Delta x)^2  \]
\[ U _{g} = mgh \]
\[ \text{Above two equations found using: } \Delta U=-W \]
\[ \text{Only true for conservative forces}  \]
\[ \text{Also: } W = \Delta KE \]

\[ \text{When we have \underline{only} conservative forces:}  \]
\[ W _{tot} = \Delta KE \]
\[ W _{tot} = - \Delta U \]
\[ \text{Meaning that: } \Delta KE = - \Delta U \]

\[ KE _{f} - KE _{o} = - (U _{f} - U _{o} )\]
\[ KE _{f}  + U _{f} = KE _{o}  + U _{o}  \]
{This is the conservation of energy}

\[ \text{Let's say at Point A, all energy is kinetic}  \]
\[ \text{At Point B, all energy is potential}  \]
\[ \text{At Point C, kinetic energy and potential energy are equal}  \]

What's the velocity at Point C?
\[ E _{totc} = KE _{c} + U _{c}  \]
\[ KE _{c} = \frac{1}{2} mv_c^2 \]
\[ \text{Then, } KE_c= \frac{1}{2} mv_c ^2 = \frac{1}{2} E _{tot}  = \frac{1}{2} \left[ \frac{1}{2} mv _{a} ^2  \right]\]
\[ \text{Therefore, } v_c = \frac{v_a}{\sqrt{2} }  \]

Consider a spring:
\begin{itemize}
	\item $\mu \ne 0$
	\item At point o, spring is fully extended
	\item At point A, spring is at equilibrium
\end{itemize}

We know that at Point o, $KE_o=0$ (Fully extended)

We know that at point A, $U_{sA}=0$ (Equilibrium)

When $\mu=0$, $KE_o + U_{so} = KE_A + U_{sA}$

With friction, $KE_A$ will be less because $W_f$ is taking energy out of the system

\[ W_f+U _{so} =KE_A \]
Place work from non-conservative forces with the initial energy of the system (because the work from friction is going to be negative)
\[ KE_o + U_o + W _{\text{nc}} =KE_f + U_f \]


\end{document}
