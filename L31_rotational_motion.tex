\documentclass[fleqn]{article}
\setlength{\parskip}{\baselineskip}%
\setlength{\parindent}{0pt}%

\usepackage[table]{xcolor}
\usepackage{siunitx}
\usepackage{tabularx}
\usepackage{float}
\usepackage{amsmath}

\begin{document}
\setlength{\mathindent}{0pt}
\section*{Rotational Motion}
Still assuming $R$ is constant.

The angle $\theta$ makes an arc: $\Delta s$
\[ \Delta s=R \cdot \theta \]
\[ v = \frac{d \Delta s}{dt} = R \frac{d \theta}{dt} \]

Remember: $\frac{d \theta}{dt} = \omega$

If radian $= \Delta s = R$ then $\theta = 1$ radian, and $\theta = \frac{\Delta s}{R} $

$v$ (tangential velocity) $= R \omega$
\[ a = \frac{dv}{dt} = R \frac{d \omega}{dt}  \]
\[ \text{We're going to call } \frac{d \omega}{dt}\text{: } \alpha \text{ (angular acceleration)}  \]
Assume $\alpha$ is constant

Therefore, \underline{$\vec{a}_{t}$ (tangential acceleration) = $R \alpha$}

If we take:
\[ \omega = \frac{d \theta}{dt}  \]
\[ d \theta = w\, dt \]
\[ \int d \theta  = \int \omega\, dt\]

Using:
\[ \alpha = \frac{d \omega}{dt} \]
\[ d \omega = \alpha \, dt \]
\[ \int d \omega = \int \alpha \, dt \]
\[ \Delta \omega = \alpha t \]
\[ \omega = \omega _{0} + \alpha t  \]

\[ \int d \theta  = \int \omega\, dt\]
\[ \omega = \omega _{0} + \alpha t  \]
\[ \Delta \theta = \int \left( \omega _{0} + \alpha t  \right)\, dt
	= \omega _{0}t + \frac{1}{2} \alpha t^2
\]

\underline{Angular Kinematics}
\begin{enumerate}
	\item $\vec{\theta} = \vec{\theta }_{0} + \vec{\omega }_{0}t + \frac{1}{2} \vec{\alpha }t^2    $
	\item $\vec{\omega } = \vec{\omega} _{0} + \vec{\alpha}t    $
	\item $\omega ^2 = {\omega _{0}}^2 + 2 \alpha \Delta \theta $
\end{enumerate}
\end{document}
