\documentclass[fleqn]{article}
\setlength{\parskip}{\baselineskip}%
\setlength{\parindent}{0pt}%

\usepackage[table]{xcolor}
\usepackage{siunitx}
\usepackage{tabularx}
\usepackage{float}
\usepackage{amsmath}

\begin{document}
\setlength{\mathindent}{0pt}
\section*{Examples of Periodic Motion}
\[ \boxed {x(t) = A \sin \left( \omega t + \phi \right) } \]

A: amplitude: $\Delta x _{\text{max} } $

$\omega$: angular velocity: $\omega = \frac{2 \pi}{T} = \sqrt{\frac{k}{m} } = 2 \pi f$

$\phi$: phase shift: $\Delta x(t=0) = x_0 = A \sin \phi$

When $\Delta x=A$, $\phi = \frac{\pi}{2} $

When $\Delta x=0$, $\phi = 0, \pi$

Should be $\phi = \pi$ because negative motion

\[ v=A \omega \cos \left( \omega t + \phi \right) \]
\[ a = A \omega ^2 \left[ - \sin \left( \omega t + \phi \right) \right] \]

\[ v _{\text{max} } = A \omega (1) \]
\[ a _{\text{max} } = A \omega ^2 (1) \]

\[ E _{\text{tot} } = \frac{1}{2} mv^2 + \frac{1}{2} k \Delta x ^2 \]
\[ = \frac{1}{2} m \left[ A \omega \cos \left( \omega t + \phi \right) \right]^2 + \frac{1}{2} k \left[ A \sin \left( \omega t + \phi \right) \right]^2 \]
\[ \frac{1}{2} A^2 \left[ m \omega ^2 \cos ^2 \left( \omega t + \phi \right) + k \sin ^2 \left( \omega t + \phi \right) \right] \]
$E=$constant IF $m \omega ^2=k$, true
\[ \boxed{E _{\text{tot} }= \frac{1}{2} kA^2 = \frac{1}{2} m \omega ^2 A^2 = \frac{1}{2} m \left( v _{\text{ max} }  \right)^2} \]
\end{document}
