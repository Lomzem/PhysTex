\documentclass[fleqn]{article}
\setlength{\parskip}{\baselineskip}%
\setlength{\parindent}{0pt}%

\usepackage[table]{xcolor}
\usepackage{siunitx}
\usepackage{tabularx}
\usepackage{float}
\usepackage{amsmath}

\begin{document}
\setlength{\mathindent}{0pt}
\section*{Impulse and Momentum}
\[ \vec{F}_{2,1}=-\vec{F}_{1,2}     \]
Newton's 3rd Law, only when $\sum \vec{F}=0 $

\[ \sum \vec{F}_{2,1}=m_2a_2  \]
\[ \sum \vec{F}=m \vec{a}   \]
\[ =m \frac{d \vec{v} }{dt}  \]
\[ =\frac{d}{dt} (m \vec{v} ) \]
\[ =\frac{d}{dt} \vec{p}  \]
\[ \vec{p} =m \vec{v} = \text{momentum}  \]

\[ \int{d \vec{p} } = \int{\sum \vec{F} }\ dt\]
\[ \Delta \vec{p}= \int{\sum \vec{F} }\ dt = \text{impulse in } kg \frac{m}{s} =Ns \]
\[ \Delta \vec{p} _1  +\Delta \vec{p}_{2} = 0  \]
\[ \int_{0}^{t} \left( \vec{F} _{1,2} + \vec{F} _{2,1}  \right) dt = 0 \]

\[ \vec{p} _{o1} + \vec{p} _{o2} = \vec{p} _{f1} + \vec{p} _{f, 2} = \text{Conservation of Momentum}  \]

\end{document}




















