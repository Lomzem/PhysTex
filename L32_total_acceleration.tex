\documentclass[fleqn]{article}
\setlength{\parskip}{\baselineskip}%
\setlength{\parindent}{0pt}%

\usepackage[table]{xcolor}
\usepackage{siunitx}
\usepackage{tabularx}
\usepackage{float}
\usepackage{amsmath}

\begin{document}
\setlength{\mathindent}{0pt}
\section*{Total Acceleration}
\[ \vec{a} _{t} = R \alpha \text{, in the direction tangent the circle}  \]
\[ \text{\underline{Important:} If we assume } R \text{ and } \alpha \text{ are constant, } |\vec{a}_{t}| \text{ will be \underline{constant}.}       \]

\[ \text{There's also } \vec{a} _{c} \text{ that always points to the center}     \]
\[ \vec{a}_{c =} \frac{v^2}{R} = R \omega ^2 \left( - \hat{r} \right)   \]
\[ \hat{r} \text{ means towards the center of the circle}  \]
\[ \vec{a}_{c} \text{ and } \vec{a}_{t} \text{ combine to get } \vec{a}_{\text{tot} }         \]

\[ \text{By definition, } \vec{a}_{c} \text{ is \underline{always} tangential to } \vec{a} _{t}      \]
\[ a _{\text{tot} } = \sqrt{{a_c}^2 + {a_t}^2}  \]
\[ \tan \theta = \frac{a_c}{a_t}  \]
\end{document}
