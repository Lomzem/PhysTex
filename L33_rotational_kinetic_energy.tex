\documentclass[fleqn]{article}
\setlength{\parskip}{\baselineskip}%
\setlength{\parindent}{0pt}%

\usepackage[table]{xcolor}
\usepackage{siunitx}
\usepackage{tabularx}
\usepackage{float}
\usepackage{amsmath}

\begin{document}
\setlength{\mathindent}{0pt}
\section*{Rotational Kinetic Energy}
\[ KE = \text{?}  \]
\[ KE = \frac{1}{2} mv^2 \]
\[ v = R \omega \]
\[ KE = \frac{1}{2} m \left( R \omega \right)^2 \]
\[ KE = \frac{1}{2} \left( m R^2 \right) \omega ^2 \]
\[ KE = \frac{1}{2} I \omega ^2 \text{, I: moment of intertia, } I = mR^2 \]
\[ I _{pp} = m R^2 \text{, pp = point particle, moment of intertia for a point particle}   \]
Earth is a point particle going around the sun. The Earth isn't one when we're standing on it. Point particle depends on the scale that we're looking at it.

"Rigid body" means $\omega _{1} = \omega _{2}  $.
Planets are not a rigid body because each planet has a different orbital period.

\[ KE _{tot} =  \frac{1}{2} \left( m_1 {R_1}^2 \right) \omega ^2 + \frac{1}{2} \left( m_2{R_2}^2 \right) \omega ^2 \]
\[ \frac{1}{2} \left[ m_1{R_1} ^2 + m_2{R_2} ^2  \right] \omega ^2 \]
Brackets = Moment of intertia for both objects
\[ KE _{tot} = \frac{1}{2} I _{tot} \omega ^2  \]
\[ I _{tot} = \sum^N_{i=1} I _{i}   \]
\[ I = \sum^{10}_{i=1} m_i {r_i}^2 \]
As $n \to \infty$, $m \to dm$
\[ \int_{0}^{M} dmr^2\  \]
Don't know how mass relates to distance, so can't integrate

linear density: $\lambda = \frac{dm}{dr} $, $\lambda = $ lambda

How does a small change in mass relate to a small chance in distance?

$dm = \lambda dr$

Assume: $\lambda $ is constant

\[ I = \int_{0}^{L}{\lambda dr\, r^2}\  \]
\[ I = \lambda \int_{0}^{L} r^2 dr\ \]
\[ I = \frac{\lambda L^3}{3}  \]
\[ \lambda = \text{const.} = \frac{M}{L}   \]
\[ I = \frac{1}{3} ML^2 \]

Rotating about center:
\[ I = \int_{- \frac{L}{2} }^{\frac{L}{2} } \lambda r^2 \ dr \]
\[ I = 2 \int_{0}^{\frac{L}{2} } \lambda r^2\ dr \]
\[ I = \left[ \frac{\lambda r^3}{3} \right]^{L/2}_{-L/2} = \frac{\lambda}{3} \left[ \frac{L^3}{8} - \frac{-L^3}{8}  \right]  \]
\[ I = \frac{\lambda}{3} \frac{L^3}{4}  \]
\[ I = \frac{1}{12} ML^2 \]

\end{document}























