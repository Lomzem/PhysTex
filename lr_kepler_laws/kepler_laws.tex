\documentclass[fleqn]{article}
\setlength{\parskip}{\baselineskip}%
\setlength{\parindent}{0pt}%

\usepackage[table]{xcolor}
\usepackage{siunitx}
\usepackage{tabularx}
\usepackage{float}
\usepackage{amsmath}
\usepackage{graphicx}

\title{Kepler's Laws Lab Report}
\author{Lawjay Lee}
\date{}

\begin{document}
\maketitle

\section*{Lab Partners:}
\begin{itemize}
	\item Raphael H.
	\item Ishmum N.
	\item Wyatt S.
\end{itemize}

\section*{Introduction}
In this lab, our group wanted to verify Kepler's three laws of astronomy.

First, we created an ellipse with a spark timer and recording paper, measured its perihelion and aphelion, and determined the focal points to verify Kepler's first law.

Second, we used the same recording paper and calculated the area enclosed by two adjacent dots. We repeated the area calculations to get six total areas. We calculated whether these areas were equal to each other to verify Kepler's second law.

Finally, we used periods of revolutions and mean distances from observations of different systems to verify Kepler's third law.
\pagebreak
\section*{Raw Data}
\subsubsection*{\underline{Part 1}}
Measurements of ellipse created from spark timer and recording paper:
\begin{table}[H]
	\setlength{\extrarowheight}{1pt}
	\begin{tabular}{|l|r|}
		\hline
		Semi-Major Axis (cm) & 13.8 \\ \hline
		Semi-Minor Axis (cm) & 6.90 \\ \hline
		% focus distance (cm)  & 11.9 \\ \hline
		% perihelion (cm)      & 1.60 \\ \hline
		% aphelion (cm)        & 25.9 \\ \hline
	\end{tabular}
\end{table}

\subsubsection*{\underline{Part 2}}
Measurements of triangles formed from two adjacent points on recording paper:
\begin{table}[H]
	\setlength{\extrarowheight}{1pt}
	\begin{tabular}{|l|r|r|}
		\hline
		Triangle & Height (cm) & Base (cm) \\ \hline
		1        & 1.40        & 13.3      \\ \hline
		2        & 1.25        & 13.6      \\ \hline
		3        & 1.90        & 13.0      \\ \hline
		4        & 1.25        & 13.6      \\ \hline
		5        & 1.65        & 13.1      \\ \hline
		6        & 2.85        & 7.30      \\ \hline
	\end{tabular}
\end{table}

\subsubsection*{\underline{Part 3}}
\begin{table}[H]
	\setlength{\extrarowheight}{1pt}
	\begin{tabular}{|l|r|r|}
		\hline
		Trial & Num Dots & Time/Dot (s) \\ \hline
		1     & 27       & 0.0300       \\ \hline
		2     & 18       & 0.0200       \\ \hline
		3     & 15       & 0.0400       \\ \hline
		4     & 15       & 0.0800       \\ \hline
	\end{tabular}
\end{table}

\section*{Data Analysis}
\subsubsection*{\underline{Part 1}}
\begin{figure}[H]
	\includegraphics[width=0.75\textwidth]{part1_overview.png}
\end{figure}
\[ f = \sqrt{R^2 - b^2}  \]
\[ f = \sqrt{(13.8 \text{ cm} )^2 - (6.90 \text{ cm} )^2}\,  \text{, numbers from raw data}  \]
\[ f = 11.9 \text{ cm}  \]

Manually measured distances using calculated focus:

perihelion = $1.60$ cm

aphelion = $25.9$ cm

Does perihelion $+$ aphelion $=$ $2 ($semi-major axis$)$ ?
\[ 1.60 \text{ cm} + 25.9 \text{ cm} = 2(13.8 \text{ cm} ) \]
\[ 27.5 \text{ cm} = 27.6 \text{ cm}  \]
\[ \% \text{ Error} = \frac{27.5 \text{ cm} - 27.6 \text{ cm} }{27.6 \text{ cm} } \times 100 \% = -0.362 \%\]

\subsubsection*{\underline{Part 2}}
\begin{table}[H]
	\setlength{\extrarowheight}{1pt}
	\begin{tabular}{|l|r|r|r|}
		\hline
		Triangle & Height (cm) & Base (cm) & Area (cm$^2$) \\ \hline
		1        & 1.40        & 13.3      & 9.31          \\ \hline
		2        & 1.25        & 13.6      & 8.50          \\ \hline
		3        & 1.90        & 13.0      & 12.4          \\ \hline
		4        & 1.25        & 13.6      & 8.50          \\ \hline
		5        & 1.65        & 13.1      & 10.8          \\ \hline
		6        & 2.85        & 7.30      & 10.4          \\ \hline
	\end{tabular}
\end{table}
Area calculated from area of a triangle: $\frac{1}{2} bh$

Stddev calculated using excel for all six triangles' areas: $1.50$ cm

\subsubsection*{\underline{Part 3}}
\begin{table}[H]
	\begin{tabular}{|l|r|r|r|r|r|r|}
		\hline
		Trial & Num Dots & Time/Dot (s) & Period (s) & R (m)  & ln(T)  & ln(R) \\ \hline
		1     & 27       & 0.03         & 0.810      & 0.0900 & -0.211 & -2.41 \\ \hline
		2     & 18       & 0.02         & 0.360      & 0.0540 & -1.022 & -2.92 \\ \hline
		3     & 15       & 0.04         & 0.600      & 0.0790 & -0.511 & -2.54 \\ \hline
		4     & 15       & 0.08         & 1.20       & 0.126  & 0.182  & -2.07 \\ \hline
	\end{tabular}
\end{table}
Period (also shown as T) found using:
\[ T =  \text{(Num Dots)(Time per Dot)}\]

R represents length of semi-major axis in elliptical system

From Kepler's 3rd Law:
\[ R^3 = k T^2 \]
\[ k = \frac{Gm_c}{4 \pi ^2}  \]
\[ R = \left(k^{1/3}\right) \left(T ^{2/3}\right) \]
\[ R = cT^n \]
\[ \ln R = \ln \left(cT^n  \right) \]
\[\ln R = n \ln \left( T \right) + \ln \left( c \right) \]

\pagebreak
Plotting our observed $\ln R$ vs. $\ln T$ into LoggerPro:
\begin{figure}[H]
	\includegraphics[width=\textwidth]{part3graph.png}
\end{figure}

Slope represents $n$ in linear relationship

$n$ represents exponent of $T$ in:
\[R = cT^n\]

Kepler's 3rd Law states $n=\frac{2}{3} $
\[ \text{\% Error of $n$} = \frac{0.688-\frac{2}{3} }{\frac{2}{3} } \times 100\% = 3.14\%\]

\subsubsection*{Finding the mass of central object:}
y-intercept represents $\ln c $
\[ c = e^{\ln c} = e ^{-2.22} = 0.109 \]
c represents $k^{\frac{1}{3} }$
\[ k = c^3 = 0.109^3 = 0.00130 \frac{\text{Nm}^2 }{kg}  \]
Solving for $m_c$ in $ k = \frac{Gm_c}{4 \pi ^2}  $ from Kepler's 3rd Law:
\[ m_c = \frac{4k \pi ^2}{G}  \]
\[ m_c = \frac{(4)(0.00130 \frac{\text{Nm}^2 }{kg} )(\pi ^2)}{6.67 \times 10^{-11} \frac{\text{Nm}^2 }{\text{kg}^2 }  }  \]
\[ m_c = 7.67 \times 10^8 \text{ kg}  \]
For comparison,
\[ m _{\text{Earth} } = 5.96 \times 10^{24} \text{ kg}  \]

The central mass is \underline{significantly} smaller.
\section*{Conclusion}
Based on our findings in all three parts, our group concluded that Kepler's Three Laws for astronomy held true for our elliptical systems.

In Part 1, we successfully found the foci in our ellipse generated from the spark timer and recording paper. Moreover, we successfully verified that the sum of the perihelion and aphelion's distances from the focal points did equal twice the semi-major axis with a percent error of $-0.362$\%.

In Part 2, we found that the areas of the "triangles" formed between two points created by the spark timer differed with a standard deviation of 1.50 cm.

Since these areas were calculated using assumptions that will be discussed later, these standard deviation seems to be small enough to conclude that Kepler's 2nd Law correctly describes how areas created in equal intervals of time are equal to each other.

In Part 3, we successfully verified Kepler's 3rd Law, which states that $T^2 \propto R^3$. Rearranging this relationship with k as the constant of proportionality:
\[ R = \left( k^{1/3 } \right) \left( T ^{2/3}  \right) \]
We found that the exponent of $T$ from our observations had a percent error of $3.14\%$. Since this percent error seems relatively small, we concluded that this relationship between the semi-major axis and the period correctly described our system.

The small percent error in Part 1 may have come from errors in precision of the ruler we used to find the measurements.

In Part 2, we used an assumption: the time intervals between the points created by the spark timer were small enough that it would allow us to find the sub-areas of the ellipse using the formula for the area of a triangle. However, we could have introduced some error in our results by making this assumption.

One way we could have improved the accuracy of the experiment is to set a higher Hz time interval and create points more frequently on the recording paper. By having smaller intervals of time, we would have more accurate sub-areas.

\subsection*{Questions}
1. Found mass central in Part 3 Data Analysis
2. \begin{figure}[H]
	\includegraphics[width=\textwidth]{question2.png}
\end{figure}
\end{document}

