\documentclass[fleqn]{article}
\setlength{\parskip}{\baselineskip}%
\setlength{\parindent}{0pt}%

\usepackage[table]{xcolor}
\usepackage{siunitx}
\usepackage{tabularx}
\usepackage{float}
\usepackage{amsmath}

\title{Kepler's Laws Lab Report}
\author{Lawjay Lee}
\date{}

\begin{document}
\maketitle

\section*{Lab Partners:}
\begin{itemize}
	\item Raphael H.
	\item Ishmum N.
	\item Wyatt S.
\end{itemize}

\section*{Introduction}
In this lab, our group wanted to verify Kepler's three laws of astronomy.

First, we created an ellipse with a spark timer and recording paper, measured its perihelion and aphelion, and determined the focal points to verify Kepler's first law.

Second, we used the same recording paper and calculated the area enclosed by two adjacent dots. We repeated the area calculations to get six total areas. We calculated whether these areas were equal to each other to verify Kepler's second law.

Finally, we used periods of revolutions and mean distances from observations of different systems to verify Kepler's third law.
\section*{Raw Data}

\section*{Data Analysis}

\section*{Conclusion}
\end{document}
